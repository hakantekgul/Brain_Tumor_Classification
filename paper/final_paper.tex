\documentclass[conference]{IEEEtran}
\IEEEoverridecommandlockouts
% The preceding line is only needed to identify funding in the first footnote. If that is unneeded, please comment it out.
\usepackage{cite}
\usepackage{amsmath,amssymb,amsfonts}
\usepackage{algorithmic}
\usepackage{graphicx}
\usepackage{textcomp}
\usepackage{xcolor}
\def\BibTeX{{\rm B\kern-.05em{\sc i\kern-.025em b}\kern-.08em
    T\kern-.1667em\lower.7ex\hbox{E}\kern-.125emX}}
\begin{document}

\title{Classification of Brain Tumors by Image Processing and Ensemble Learning\\}

\author{\IEEEauthorblockN{1\textsuperscript{st} Hakan Tekgul}
\IEEEauthorblockA{\textit{Electrical and Computer Engineering} \\
\textit{Georgia Institute of Technology}\\
Metz, France \\
htekgul3@gatech.edu}
\and
\IEEEauthorblockN{2\textsuperscript{nd} Jad Sinno}
\IEEEauthorblockA{\textit{Electrical and Computer Engineering} \\
\textit{Georgia Institute of Technology}\\
Metz, France \\
jsinno3@gatech.edu}
\and
\IEEEauthorblockN{3\textsuperscript{rd} Madhur Gupta}
\IEEEauthorblockA{\textit{Electrical and Computer Engineering} \\
\textit{Georgia Institute of Technology}\\
Metz, France \\
mgupta325@gatech.edu}
}

\maketitle

\begin{abstract}
When diagnosed at a late stage, brain cancers and tumors provide very short life expectancy and a high mortality rate because of their aggressive nature. In order to diagnose and detect the tumor on the brain at the earliest stage possible, various medical imaging techniques such as Computed Tomography (CT) or Magnetic Resonance Imaging (MRI) are used. Unfortunately, the diagnosis and treatment of the tumor depends highly on the physician’s medical knowledge. Hence, many different systems have been proposed in the past decade to automatically detect brain tumors by using machine learning. One major problem with these approaches is the fact that they usually need lots of training data (MRI Scans).  

In this project, we detect brain tumors from MRI images with a limited amount of training data, around 120 images. Specifically, we use an ensemble learning approach that is based on Support Vector Machines (SVM). After applying image processing techniques such as low-pass filtering to all the images, the most common prediction from different SVM classifiers is chosen for a testing image. Validation accuracies are used to select the best image processing techniques and their optimal parameters. Our experimental results suggest that the Ensemble Classifier achieves an accuracy of 87\% with low time complexity.
\end{abstract}

\begin{IEEEkeywords}
image classification, ensemble learning, medical imaging, SVM
\end{IEEEkeywords}

\section{Introduction}\label{intro}
\subsection{Motivation}
% Explain the motivation behind project and goal of project
% Add graphs for increased number of brain tumors etc. 

\subsection{Related Work} 
% Discuss some related work. Provide around 5-6 projects about tumor classification and summarize their improvements/results
% DO NOT FORGET TO CITE

\section{Methodology}\label{method}
% Explain our methodology behind the project
% Put the flowchart here 

% Talk briefly about reizing and grayscale. Maybe put the graph on image sizes? 
\subsection{Image Preprocessing}
% Describe the equalization and contrast images

\subsection{Image Filtering}
% Describe and explain the different filters 

\subsection{Machine Learning}
% Briefly describe the machine learning algorithms 
% Cross validation
% Learning and validation curves

\section{Experimental Results}\label{results}
% Provide experimental results here
% Provide 3 different tables --> cluster all k-NN results together into one table, same for SVM and NN
% Final Confusion Matrix 

\section{Analysis of Results}\label{analysis}
% Detailed analysis of all results and process
% Analyze the results for each filter, different sigma values, and for each ML algorithm
% Which image filter worked the best? 
% Which algorithm worked the best?
% How helpful was ensemble learning?
% Discuss false negatives and accuracy 

\section{Conclusion and Future Work}\label{conclusion}
% Briefly summarize main ideas of project
% Discuss future work such as tumor segmentation

\section*{Acknowledgment}

The preferred spelling of the word ``acknowledgment'' in America is without 
an ``e'' after the ``g''. Avoid the stilted expression ``one of us (R. B. 
G.) thanks $\ldots$''. Instead, try ``R. B. G. thanks$\ldots$''. Put sponsor 
acknowledgments in the unnumbered footnote on the first page.

\section*{References}

Please number citations consecutively within brackets \cite{b1}. The 
sentence punctuation follows the bracket \cite{b2}. Refer simply to the reference 
number, as in \cite{b3}---do not use ``Ref. \cite{b3}'' or ``reference \cite{b3}'' except at 
the beginning of a sentence: ``Reference \cite{b3} was the first $\ldots$''

Number footnotes separately in superscripts. Place the actual footnote at 
the bottom of the column in which it was cited. Do not put footnotes in the 
abstract or reference list. Use letters for table footnotes.

Unless there are six authors or more give all authors' names; do not use 
``et al.''. Papers that have not been published, even if they have been 
submitted for publication, should be cited as ``unpublished'' \cite{b4}. Papers 
that have been accepted for publication should be cited as ``in press'' \cite{b5}. 
Capitalize only the first word in a paper title, except for proper nouns and 
element symbols.

For papers published in translation journals, please give the English 
citation first, followed by the original foreign-language citation \cite{b6}.

\begin{thebibliography}{00}
\bibitem{b1} G. Eason, B. Noble, and I. N. Sneddon, ``On certain integrals of Lipschitz-Hankel type involving products of Bessel functions,'' Phil. Trans. Roy. Soc. London, vol. A247, pp. 529--551, April 1955.
\bibitem{b2} J. Clerk Maxwell, A Treatise on Electricity and Magnetism, 3rd ed., vol. 2. Oxford: Clarendon, 1892, pp.68--73.
\bibitem{b3} I. S. Jacobs and C. P. Bean, ``Fine particles, thin films and exchange anisotropy,'' in Magnetism, vol. III, G. T. Rado and H. Suhl, Eds. New York: Academic, 1963, pp. 271--350.
\bibitem{b4} K. Elissa, ``Title of paper if known,'' unpublished.
\bibitem{b5} R. Nicole, ``Title of paper with only first word capitalized,'' J. Name Stand. Abbrev., in press.
\bibitem{b6} Y. Yorozu, M. Hirano, K. Oka, and Y. Tagawa, ``Electron spectroscopy studies on magneto-optical media and plastic substrate interface,'' IEEE Transl. J. Magn. Japan, vol. 2, pp. 740--741, August 1987 [Digests 9th Annual Conf. Magnetics Japan, p. 301, 1982].
\bibitem{b7} M. Young, The Technical Writer's Handbook. Mill Valley, CA: University Science, 1989.
\end{thebibliography}

\end{document}
